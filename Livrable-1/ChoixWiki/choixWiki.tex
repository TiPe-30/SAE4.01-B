\documentclass[../Livrable1.tex]{subfiles}
\usepackage{tabularx}
\usepackage{multicol}
\usepackage{ragged2e}

\begin{document}
	
\justify{
Dans le cadre de la mise en oeuvre de cette infrastructure, nous devrons concevoir un wiki détaillé expliquant à un intervenant extérieur comment fonctionnera cette infrastructure réseau afin qu'il ou elle puisse être en mesure d'administrer ce réseau.
}

\begin{itemize}
    \item Critères de Choix :
    \begin{multicols}{2}
    \begin{itemize}
    	\item Facilité d'utilisation
    	\item Fonctionnalités
    	\item Gestion des utilisateurs
    	\item Sécurité
    	\item Cout
    	\item Support et communauté
    \end{itemize}
	\end{multicols}

    \item Comparaison des solutions :
    	
    	\begin{table}[h]
		\centering
    	\renewcommand{\arraystretch}{1.5}
    	\begin{tabularx}{1.3\textwidth}{|X|X|X|}
    		\hline
    		Wiki & Avantages & Inconvénients \\ 
    		\hline
    		\textbf{DokuWiki} & Léger, pas besoin de base de données,  bon contrôle des accès, support LDAP. & Interface de base au design moins moderne. \\
    		\hline
    		\textbf{MediaWiki} & Utilisé par Wikipedia, puissant, support des extensions, bonne gestion des droits. & Configuration plus lourde, nécessite MySQL/PostgreSQL. \\
    		\hline
    		\textbf{Wiki.Js} & Moderne, supporte Markdown, bonne intégration LDAP/SSO, interface intuitive. & Plus lourd, nécessite une base de données et Node.js. \\
    		\hline
    		\textbf{XWiki} & Très complet, bon support des macros et extensions. & Installation plus complexe. \\
    		\hline
    \end{tabularx}
     \end{table}
\end{itemize}



\justify{
	Moteur de Wiki choisi : \textbf{Wiki.Js} \newline
	\begin{itemize}
		\item Interface intuitive
		\item Support de plusieurs méthodes d'authentification (dont LDAP et SAML 2.0) 
		\item Licence Gratuit : Open-source sous licence AGPL
		\item Gestion avancée des utilisateurs
		\item Facilement installable et déployable sur de nombreuses plateformes serveur et support d'un SGBD déjà maîtrisé : PostgreSQL
		\item Beaucoup de fonctionnalités incluses
		\item Programmé contre les failles de sécurité avec possibilité de faire des retours en cas de vulnérabilités constatées
	\end{itemize}
	
	
}


\end{document}