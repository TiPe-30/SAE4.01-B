\documentclass[../file_doc_script.tex]{subfiles}
\usepackage[french]{babel}
\usepackage{listings}
\usepackage{subfiles}
\usepackage{graphicx}
\usepackage[utf8]{inputenc}
\usepackage{fancyhdr}
\usepackage{tabularx}
\usepackage{listings}
\usepackage{float}
\usepackage{xcolor}
\lstset{inputencoding=utf8}

\begin{document}

Tous les scripts dévellopés dans le cadre de ce projet possède une documentation,
qui est l'éuivalent d'un manuel. Pour y avoir accès il suffit de taper le nom du script
tout seul et une documentation s'affichera.

\subsection{Utilisation du script dans un terminal}

\begin{figure}[h]
    \centering
    \includegraphics[width=1\textwidth]{../Images/doc-script-terminal.png}
    \caption{Manuel d'utilisation du script}
    \label{fig:solution1}
\end{figure}

\subsection{Implémentation de cette documentation}
Pour cette documentation, nous avons inclus un if pour détecter si aucun paramètre n'a été 
donné au script. Dans ces cas là il lève une erreur en montrant les options que peut utiliser 
l'administrateur pour ce script.

\lstinputlisting[firstline=7,lastline=21,language=bash, caption=test HTTP/HTTPS]{../ping_test.sh}

\subsection{Expression régulière}
Afin de faciliter la gestion des erreurs par les utilisateurs du script, nous avons 
mis en place une vérification des paramètres données par l'utilisateur via les regex.
Ces regex permettent de vérifier les entrées utilisateurs et de vérifier que celle ci soient 
valides et n'engendre pas de bug.

\lstinputlisting[firstline=22,lastline=23,language=bash, caption=expression régulière pour des adresses IP]{../ping_test.sh}

\end{document}
