\documentclass[../livrable-4.tex]{subfiles}
\usepackage[french]{babel}
\usepackage{listings}
\usepackage{subfiles}
\usepackage{graphicx}
\usepackage[utf8]{inputenc}
\usepackage{fancyhdr}
\usepackage{tabularx}
\usepackage{float}
\usepackage{xcolor}
\lstset{inputencoding=utf8}

\begin{document}
Dans ce document, nous allons traiter tous les scripts relatifs à l'installation et 
la configuration de l'infrastructure. Le but est ici, d'automatiser certaines tâches récurrentes
comme la configuration DHCP, les ajouts d'utilisateur au système ou au wiki etc... 

\subsection{Documentation des scripts}
Comme dans le document pour les scripts de test, un manuel pour utiliser les scripts
ont été intégré au script eux-mêmes.

\UseRawInputEncoding
\lstinputlisting[firstline=3,lastline=21,language=bash, caption=test HTTP/HTTPS]{../install_dhcp.sh}



\end{document}