\documentclass[../file.tex]{subfiles}
\usepackage{graphics}


\begin{document}
En charge de l'IT pour l'entreprise PLETER, spécialisée dans l'animation 3D, un secteur fortement dépendant de l'informatique. Il est essentiel de mettre en place une infrastructure réseau robuste et sécurisée, capable de répondre à tous les besoins de l'entreprise. Cette infrastructure doit également être conçue pour supporter l'ensemble des logiciels et systèmes d'exploitation nécessaires aux équipes créatives et de développement, afin de garantir une compatibilité optimale avec les outils utilisés (ex : Blender, Adobe Creative Cloud, logiciel Autodesk, Unreal Engine, Cinema 4D, etc).
\newline
Lors de la phase de réflexion du projet, nous avons donc décidé d'utilser proxmox afin de virtualiser
nos machines virtuelles. Le but initial
était de prendre Qemu sur le serveur interne à notre IUT(assr) afin de faire tourner une machine virtuelle
avec suffisament de ressource afin de pouvoir faire tourner notre infrastructure dedans.
Avec proxmox, comme nous avons des hyperviseurs réunit dans un cluser nous amélioront les performances tout en 
complexifiant la tâche. En effet, les interface de VLAN de proxmox rendent l'approche de configurer des routeurs 
qui permettent de connecter des VMs à travers les hyperviseurs complexe. Nous avons les VxLan de l'interface proxmox 
qui doivent ainsi contenir les hyperviseurs mais pas relier les VM à part la vm routeur qui sera relié à un hyperviseur proxmox.
D'autres approchent sont sans doute possible, mais c'est la piste sur laquelle nous cherchons.
Afin de gagner du temps sur la partie réalisation, la plupart des tâches récurrentes comme la configuration des serveurs 
DHCP qui peupleront chaque sous réseaux, ont été crées. Les test de connectivité permettent donc de réaliser les tests sans problèmes
particuliers.

\end{document}