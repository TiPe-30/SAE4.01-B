\documentclass[../file.tex]{subfiles}
\usepackage{graphics}


\begin{document}
En charge de l'IT pour l'entreprise PLETER, spécialisée dans l'animation 3D, un secteur fortement dépendant de l'informatique, il est essentiel de mettre en place une infrastructure réseau robuste et sécurisée, capable de répondre à tous les besoins de l'entreprise. Cette infrastructure doit également être conçue pour supporter l'ensemble des logiciels et systèmes d'exploitation nécessaires aux équipes créatives et de développement, afin de garantir une compatibilité optimale avec les outils utilisés (ex. : Blender, Adobe Creative Cloud, logiciels Autodesk, Unreal Engine, Cinema 4D, etc.).

\newline
Lors de la phase de réflexion du projet, nous avons décidé d'utiliser Proxmox afin de virtualiser nos machines. L'objectif initial était d'utiliser QEMU sur le serveur interne de notre IUT (ASSR) pour faire tourner une machine virtuelle disposant de suffisamment de ressources afin d'héberger notre infrastructure.

Avec Proxmox, comme nous avons des hyperviseurs réunis dans un cluster, nous améliorons les performances tout en complexifiant la tâche. En effet, les interfaces VLAN de Proxmox rendent la configuration des routeurs, qui permettent de connecter des VMs à travers les hyperviseurs, plus complexe. Nous avons les VxLAN de l'interface Proxmox, qui doivent inclure les hyperviseurs, mais sans relier directement les VMs, à l'exception de la VM routeur qui sera connectée à un hyperviseur Proxmox.

D'autres approches sont sans doute possibles, mais c'est la piste sur laquelle nous travaillons actuellement.

Afin de gagner du temps sur la partie réalisation, la plupart des tâches récurrentes, comme la configuration des serveurs DHCP qui alimenteront chaque sous-réseau, ont été automatisées. Les tests de connectivité permettent ainsi de valider le bon fonctionnement du réseau sans rencontrer de problèmes particuliers.

\end{document}
